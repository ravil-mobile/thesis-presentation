\subsubsection{Numerical Accuracy}


\begin{frame}[t]{Numerical Accuracy: I}
	\small
	\begin{itemize}
		\item Pivoting is a big issue in direct sparse methods
		\begin{itemize}
			\item analysis: absence of numerical information
			
			\item numerical factorization: it may distort all prediction made during the analysis phases i.e.
			\begin{itemize}
				\item fill-in prediction
				\item load balancing
			\end{itemize}
		\end{itemize}
	\end{itemize}

	\begin{center}
		Therefore, threshold pivoting is commonly used for direct sparse methods
	\end{center}

	Threshold pivoting means that a pivot $|a_{i,i}|$ is accepted if it satisfies:\\
	
	\begin{equation}\label{eq:lc-1}
		|a_{i,i}| \geq \alpha \times max_{k=i \dots n} |a_{k,i}|
	\end{equation}
	
	where $\alpha \in [0,1]$ and $k=i \dots n$ represents row indices of column $i$ within the fully summed block of a frontal matrix.\\
	
\end{frame}



\begin{frame}[t]{Numerical Accuracy: II}
	\small
	\begin{itemize}
		\item In case of small values of $\alpha$
		
		\item solutions can be numerically inaccurate
		\item may demand to perform solution refinements 
	\end{itemize}

	\vspace{5mm}
	As an example, solution accuracy can be improved using:
	\begin{itemize}
		\item  iterative refinement method based a on solution residual
		
		\item  resulting $LU$ decomposition can be used as a preconditioner for a Krylov-based method e.g. GMRES
	\end{itemize}
\end{frame}

\subsubsection{Aspects}

%%%%%%%%%%%%%%%%%%%%%%%%%%%%%%%%%%%%%%%%%%%%%%%%%%%%%%%%%%
\ifSpeech
\begin{frame}[t]{Parallelization Aspects: Speech}
    \small
    \justifying
    \begin{itemize}
    	\item Iterative methods usually make use of simple linear algebra kernels e.g. dot products, matrix-vector products, etc.
    	
    	\begin{itemize}
    		\item Therefore, sparsity of linear systems can be handled well
    	\end{itemize}
    	
    	\item The methods make use of data-based parallelism
    	
    	\begin{itemize}
    		\item vectors and matrices are distributed among multiple processing units
    		
    		\item Therefore, the methods scales well in parallel
    		
    		\item the main drop of parallel performance mainly
comes from process-communication overheads	
    	\end{itemize}
    	
    \end{itemize}   
\end{frame}

\begin{frame}[t]{Preconditioners: Speech}
\small
\justifying

\begin{itemize}
	\item In case of a "bad" initial guess $x_{0}$, convergence of iterative methods strongly depends on an involved matrix and, in particular, on its condition number
	
	\item A big condition number usually leads to slow convergence
	
	\item a linear transformation, known as preconditioning, can be applied to the original system of equations
	
	\item to reduce its condition number and, therefore, to accelerate convergence
\end{itemize}

\begin{itemize}
	\item a good preconditioning algorithm should result in
	
	\begin{itemize}
		\item low computational cost
		\item low storage space 
		\item low condition number of the transformed system
		\item adapted for parallel executions as well
	\end{itemize}
\end{itemize}
\end{frame}
\fi


\begin{frame}[t]{Aspects}
\small
\begin{block}{Parallelization Aspects}
	
	\begin{itemize}
		\item Iterative methods usually make use of simple linear algebra kernels e.g. dot products, matrix-vector products, etc.
		
		\item Therefore, the methods efficiently make use of data-based parallelism 
	\end{itemize}
\end{block}

\begin{block}{Numerical Accuracy}
	
	\begin{itemize}
		\item Convergence to the exact solution depends on a value of the  condition number of a matrix
		
		\item Usually requires a linear transformation known as preconditioning
		
		\begin{itemize}
			\item to reduce condition number and accelerate convergence
		\end{itemize}
	\end{itemize}
\end{block}
\end{frame}


\begin{frame}[t]{Preconditioners}
\tiny
\begin{table}[!t]
	\tiny
	\centering
	\begin{tabular}{|c|c|c|c|c|}
		\hline
		\begin{tabular}[c]{@{}c@{}}Package\\ name\end{tabular}     & Origin                                                   & Method                                                        & \begin{tabular}[c]{@{}c@{}}Tuning\\ parameters\end{tabular}                                                                                                                            & Comments                                                                  \\ \hline
		block Jacobi                                               & PETSc                                                    & block Jacobi                                                  & \begin{tabular}[c]{@{}c@{}}-pc\_bjacobi\_blocks\\ -sub\_pc\_type\end{tabular}                                                                                                          & -                                                                         \\ \hline
		\begin{tabular}[c]{@{}c@{}}additive\\ Schwarz\end{tabular} & PETSc                                                    & \begin{tabular}[c]{@{}c@{}}additive\\ Schwarz\end{tabular}    & \begin{tabular}[c]{@{}c@{}}-pc\_asm\_blocks\\ -pc\_asm\_overlap\\ -pc\_asm\_type\\ -pc\_asm\_local\_type\\ -sub\_pc\_type\end{tabular}                                                 & -                                                                         \\ \hline
		euclid                                                     & hypre                                                    & ILU(k)                                                        & \begin{tabular}[c]{@{}c@{}}-nlevel\\ -thresh\\ -filter\end{tabular}                                                                                                                    & \begin{tabular}[c]{@{}c@{}}deprecated\\ form\\ PETSc\end{tabular}         \\ \hline
		pilut                                                      & hypre                                                    & ILU(t)                                                        & \begin{tabular}[c]{@{}c@{}}-pc\_hypre\_pilut\_tol\\ -pc\_hypre\_pilut\_maxiter\\ -pc\_hypre\_pilut\_factorrowsize\end{tabular}                                                         & -                                                                         \\ \hline
		parasail                                                   & hypre                                                    & SPAI                                                          & \begin{tabular}[c]{@{}c@{}}-pc\_hypre\_parasails\_nlevels\\ -pc\_hypre\_parasails\_thresh\\ -pc\_hypre\_parasails\_filter\end{tabular}                                                 & -                                                                         \\ \hline
		SPAI                                                       & \begin{tabular}[c]{@{}c@{}}Grote,\\ Barnard\end{tabular} & SPAI                                                          & \begin{tabular}[c]{@{}c@{}}-pc\_spai\_epsilon\\ -pc\_spai\_nbstep\\ -pc\_spai\_max\\ -pc\_spai\_max\_new\\ -pc\_spai\_block\_size\\ -pc\_spai\_cache\_size\end{tabular}                & -                                                                         \\ \hline
		BoomerAMG                                                  & hypre                                                    & \begin{tabular}[c]{@{}c@{}}algebraic\\ multigrid\end{tabular} & \begin{tabular}[c]{@{}c@{}}-pc\_hypre\_boomeramg\_cycle\_type\\ -pc\_hypre\_boomeramg\_max\_levels\\ -pc\_hypre\_boomeramg\_max\_iter\\ -pc\_hypre\_boomeramg\_tol\\ etc.\end{tabular} & \begin{tabular}[c]{@{}c@{}}39 tuning\\ parameters\\ in total\end{tabular} \\ \hline
	\end{tabular}
	\caption{Parallel preconditioning algorithms available in PETSC}
	\label{table:preconditioners}
\end{table}

\end{frame}
